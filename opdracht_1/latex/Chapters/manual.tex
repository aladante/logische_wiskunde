\section{Introduction} \label{man-introduction}
This section will go in detail about the working, idea and logic of our expert system.

\section{application} \label{man-application}
We choose to create an application that will indentify the vehicle type based on user input.

By answering specific provided questions the expert system is able to identify the type of vehicle.
The vehicle type could be a land vehicle such as a car or bicycle or something like a boat.

By answering yes or no the system is able to assume the type of vehicle.


\section{Usage} \label{man-usage}
The application is run by providing the following command \textbf{isVehicle(X)}.
This will first check if where looking to identify a vehicle.

If the user types yes the system agrees on a abstract datatype vehicle.
This datatype has multiple sub types such as land vehicle or air vehicle. At last we have the unidentified
vehicle this can be a car or boat.
The answer will be decided by the answers that the user provides.

\newpage
\section{Logic} \label{man-logic}
In this section we will provide an explanation of our code.
The code is included with comments to provide as reference for the explanation.




\begin{lstlisting}{}
isVehicle(X) :- is_true('mode of transportation'), vehicle(X).
vehicle(airplane)  :- is_true('has wings'), air_vehicle().
vehicle(spaceCraft)  :- air_vehicle(), hasWheels().
vehicle(car)  :- is_true('has four wheels'), is_true('has enginge'), land_vehicle() .
vehicle(bike) :- is_true('has two wheels'), land_vehicle().
vehicle(cycle) :- is_true('has two wheels'), is_true('no engine'), land_vehicle().
vehicle(boat) :- is_true('has two wheels'), is_true('no engine'), boat_vehicle().

land_vehicle() :-
	is_true('drives on land').

air_vehicle() :-
	is_true('fly in air'), 
	is_true('has engine').

boat_vehicle() :-
	is_true("moves over water"),
    is_true("has no wheels").

hasWheels() :- not(is_true('has wheels')).

is_true(Q) :-
        format("~w?\n", [Q]),
        read(yes).
\end{lstlisting}



